\documentclass{article} 
\usepackage{amsmath}
\begin{document}

\section{Exercise Set 1.1}

\textbf{My answers}
\newline

2\\
	\textbf{a.} Is there an integer $n$ such that $n$ has a remainder of 2 when divided by 5, and a remainder of 3 when divided by 6 ?\\
	\textbf{b.} Does there exist a number $n$, such that if $n$  is divided by 5 the \emph{remainder} is 2 and if $n$ is divided by 6 the remainder is 3?
\newline

3. \textbf{Given any two distinct real numbers, there is a real number in between them}\\
	a. Given any two distinct real numbers a and b, there is a real number c such that c is in between a and b. $(a < c > b)$\\
	b. For any two real numbers $a, b$  there is a real number in between, $c$ such that $c$ is between a and b.
\newline

4.\textbf{ Given any real number, there is a real number that
is greater.}\\
	\textbf{a.} Given any real number $r$, there is another number $s$ such that $s$ is greater than $r$.\\
	\textbf{b.} For any real number $r$, there is a real number $s$ greater than $r$ such that $s>r$.
\newline

5.  \textbf{The reciprocal of any positive real number is positive.}\\
	\textbf{a.} Given any positive real number $r$, the reciprocal of $r$ is positive.\\
	\textbf{b}. For any real number $r$, if $r$ is positive, then its reciprocal is positive.\\
	\textbf{c.} If a real nubmer $r$ is positive, then the reciprocal of $r$ is positive.
\newline

6. \textbf{The cube root of any negative real number is negative.}\\
	a. Given any negative real number $s$, the cube root of $s$ is negative.\\
	b. For any real number $s$, if $s$ is negative, then the cube root of $s$ is negative.\\
	c. If a real number $s$ is negative, then $s$ cube root is negative.
\newline

7.\textbf{ Rewrite the following statements less formally, without using variables. Determine, as best as you can, whether the statements are true or false.}
	a. False, if $u < v$, then $u + v > u - v$\\
	b. True\\
	c. True, take for example $n = 1$, $1^2 = 1 \geq 1$\\
	d. True.
\newline

\end{document} 
